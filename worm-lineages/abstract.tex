\documentclass[12pt]{article}
\usepackage{amsmath}
\usepackage[letterpaper,margin=.5in]{geometry}
\usepackage{url}
\usepackage{graphicx}
\usepackage[utf8]{inputenc}
\usepackage[T1]{fontenc}
\usepackage{textcomp}
\usepackage[scaled]{helvet}
\renewcommand{\familydefault}{\sfdefault}
\pagenumbering{gobble}

\title{Cell Lineage Tracing by Whole Genome Sequencing}
\date{}
% \date{\today}
% \author{Adam Orr}

\begin{document}
\maketitle

% 250 words

% The abstract should include:
%  * a brief background of the project;
%  * specific aims, objectives, or hypotheses;
%  * the significance of the proposed research and relevance to public health;
%  * the unique features and innovation of the project;
%  * the methodology (action steps) to be used;
%  * expected results; and
%  * description of how your results will affect other research areas.
% Suggestions
%  * Be complete, but brief.
%  * Use all the space allotted.
%  * Avoid describing past accomplishments and the use of the first person.
%  * Write the abstract last so that it reflects the entire application.



\section{Abstract}

Cell lineage tracing is the practice of determining how each cell in an adult organism arises from the single progenitor zygotic cell.
The practice is commonly used in stem cell studies and studies of cancer heterogeneity.
Usually, tracing is performed by direct observation of a marked cell and its decendants.
This is a time-consuming and expensive process that cannot be performed in a high-throughput setting.
There is interest in sequencing somatic variants in decendant cells to infer cell lineages in environments that make direct observation intractable, such as in humans.
Some successful approaches in mouse use microsatellite markers, but such methods are not generally applicable.
For example, the number of microsatellites in the \textit{C. elegans} genome is likely not sufficient to reliably produce a cell fate map with microsatellite sequencing.
However, the detailed \textit{C. elegans} cell fate map provides an opportunity to create a generic method for reconstructing cell lineages via whole genome sequencing. 
\textbf{Aim 1} of the project is to develop a method of inferring cell fate maps from whole-genome sequencing data.
The method will incorporate technical sequencing replicates to increase the accuracy of mutation calls.
Several tree construction models will be compared to determine the optimal model for constructing trees of somatic variants.
\textbf{Aim 2} of the project is to test the hypothesis that cell lineage trees derived from whole genome sequencing data more closely correlate to known cell fate maps than do those inferred by microsatellites alone.
To test this hypothesis, cells isolated from transgenic \textit{C. elegans} that fluoresce in a tissue-specific manner will be separated by Fluorescence-Activated Cell Sorting and sequenced.
The method developed for Aim 1 will then be used to generate a cell fate map.
The cell fate map will then be compared to one generated solely from microsatellites and the true cell fate map.
These results will improve the understanding of how somatic mutations arise and will impact the fields of immunology and cancer biology.


\end{document}