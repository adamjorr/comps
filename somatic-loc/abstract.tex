\documentclass[12pt]{article}
\usepackage{amsmath}
\usepackage[letterpaper,margin=.5in]{geometry}
\usepackage{url}
\usepackage{graphicx}
\usepackage[utf8]{inputenc}
\usepackage[T1]{fontenc}
\usepackage{textcomp}
\usepackage[scaled]{helvet}
\renewcommand{\familydefault}{\sfdefault}
\pagenumbering{gobble}

\title{Investigating the Influence of Chromatin on the Distribution of Somatic Mutations}
\date{}
% \date{\today}
% \author{Adam Orr}

\begin{document}
\maketitle

% 250 words

% The abstract should include:
%  * a brief background of the project;
%  * specific aims, objectives, or hypotheses;
%  * the significance of the proposed research and relevance to public health;
%  * the unique features and innovation of the project;
%  * the methodology (action steps) to be used;
%  * expected results; and
%  * description of how your results will affect other research areas.
% Suggestions
%  * Be complete, but brief.
%  * Use all the space allotted.
%  * Avoid describing past accomplishments and the use of the first person.
%  * Write the abstract last so that it reflects the entire application.



\section{Abstract}
Cancer is a devastating disease caused by mutations in somatic cells that cause those cells to rapidly proliferate and evade the immune system. Despite this fact, little is known about somatic mutations themselves.
Of particular interest are how the differences between somatic tissues contribute to differences in cancer rates between tissues.
A potential explanation for this observation is the differential epigenetic states of different tissues.
This project will explore how chromatin levels in different tissues affect the distribution of somatic mutations.
\textbf{Aim 1} of this project is to test the hypothesis that the tissue somatic mutation rate of different mice tissues is correlated with the level of chromatin accessability of different genomic regions.
This will be tested by applying whole-genome sequencing and an ATAC-seq protocol to samples of various mouse tissue from a wild-type mouse and a mouse model of cancer.
A novel bioinformatic pipeline will then be developed to detect mutations and compare mutation density with chromatin density.
\textbf{Aim 2} of this project is to test the hypothesis that the distribution of mutations is significantly altered in the tissue destined to become cancerous, but not in the other tissues of the cancer model.
This hypothesis will be tested using the mutations and locations found for Aim 1.
The distribution of mutations in each tissue will be compared to the distribution of mutations in the wild-type genotype.
These results will improve the understanding of how somatic mutations arise and will impact the work of cancer and developmental biologists.



\end{document}