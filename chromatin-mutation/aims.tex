%1 page
% State concisely the goals of the proposed research and summarize the expected outcome(s),
% including the impact that the results of the proposed research will have on the research field(s)
% involved.
% List succinctly the specific objectives of the research proposed (e.g., to test a stated hypothesis,
% create a novel design, solve a specific problem, challenge an existing paradigm or clinical practice,
% address a critical barrier to progress in the field, or develop new technology)

Somatic mutations are mutations that occur in the soma.
Though they occur infrequently, the size of mammalian genomes suggest approximately one somatic mutation occurs every time a cell divides.
As these mutations are not heritable, they usually do not impact the organism harboring these mutations.
However, the somatic mutation theory of cancer suggests that somatic cells accumulate mutations until a sufficient number of mutations in oncogenes or tumor suppressor genes cause the mutated cell to become cancerous.
Thus, cancer incidence can be impacted by increasing the rate of mutation \textit{or} increasing the probability of mutation in an oncogene or tumor suppressor gene.
While the role of nonspecific mutagens in oncogenesis has been thoroughly studied, the role of factors that could change the distribution of somatic mutations has not.

%talk about clues in chromatin association and differences in tissue chromatin str.
Recent work has revealed that the locations of mutations in cancer tumors are associated with regions of high chromatin accessibility.
To fit in the nucleus, eukaryotic DNA is bound to histones and packaged into chromatin.
The DNA may be in a loosely bound, high accessibility state to allow DNA-binding factors to access the DNA.
The DNA may also be in a tightly bound, low accessibility state that restricts the ability of proteins to access the DNA.
Levels of chromatin accessibility vary with levels of gene regulation; low accessibility is associated with repressed genes and high accessibility is associated with actively expressed genes.
One of the most important sources of variability in chromatin structure is tissue differentiation.
Different tissues express particular genes and repress others by decreasing DNA accessibility.
If chromatin accessibility affects the location of mutations, and locations with high levels of accessibility vary by tissue, the locations of somatic mutations could vary in different tissues.
If there is variation in chromatin accessibility at oncogenes and tumor suppressor genes, this variation could cause differential risk of cancer for different tissues.
This hypothesis is consistent with the observation that different tissues have different rates of cancer.
However, it is still unclear whether these associations form only after the tissue becomes cancerous. 

A significant impediment to research in this area is a lack of computational methods for detecting somatic mutations in healthy tissue.
Current methods for detecting somatic mutation focus on detecting somatic mutations present in tumors, and rely on data from matched normal tissue to improve accuracy.
Cancer genotypes are significantly altered compared to their normal counterparts, and are characterized by large genomic aberrations such as chromosome rearrangements, changes in ploidy number, inversions, and insertion/deletions.
On the other hand, mutations in healthy somatic tissue are more subtle and are more likely to contain single nucleotide changes and short changes in microsatellite length.
At the same time, sequencing errors most likely appear as single nucleotide polymorphisms or short indels---indistinguishable from the types of mutations most common in healthy somatic tissue.
Additionally, the sequencing error rate is significantly higher than the somatic mutation rate of healthy cells while being lower than mutation rates of cancer tumors.
Due to the significant differences in the type and number of somatic mutations between healthy tissue and cancer tumors, current methods are insufficient for detecting somatic mutations in healthy tissue.

\textbf{Aim 1 of this project is to develop a novel bioinformatic pipeline to detect somatic mutations and estimate chromatin accessibility across the genome in somatic samples.}
This will be done by modeling and correcting sequencing errors in Next-Generation Sequencing reads to reduce erroneous genotype calls. Genotypes will then be called using model parameters that closely resemble those expected in healthy somatic tissue.
Additionally, reads generated by an ATAC-seq or similar sequence-based accessibility assay will incorporated into the alignment to improve error correction and genotype calling while providing information on chromatin accessibility.

\textbf{Aim 2 of this project is to test the hypothesis that chromatin accessibility significantly impacts mutation rate.}
The method developed in Aim 1 will be applied to several somatic samples from two mouse models of cancer.
As a control, it will also be applied to samples of the same tissues in a mouse with a similar, unmanipulated genetic background.
This will provide an estimate of the distribution of chromatin accessibility values at mutated sites; if this distribution is not significantly different between different tissues, the hypothesis is supported.

This project will provide a novel method for analyzing somatic mutations in healthy tissues and fill a knowledge gap regarding the relationship between chromatin structure and mutation rate. It will provide insight into an important chicken-and-egg problem in cancer genomics and provide an important resource for the fields of developmental biology and epigenomics.
