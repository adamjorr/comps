% !TEX program=lualatex
\RequirePackage{luatex85}
\documentclass[11pt]{article}
\usepackage{amsmath}
\usepackage[letterpaper,margin=.5in]{geometry}
\usepackage{url}
\usepackage{graphicx}
\usepackage[utf8]{inputenc}
\usepackage[T1]{fontenc}
\usepackage{textcomp}
\usepackage[scaled]{helvet}
\usepackage[compact,tiny]{titlesec}
\usepackage[style=nature, citestyle=authoryear]{biblatex}
\usepackage{tikz}
\usepackage{wrapfig}
\renewcommand{\familydefault}{\sfdefault}
\pagenumbering{gobble}
\bibliography{comps.bib}

%lung, colon, and stomach?

\begin{document}
% \section{Title}
% \maketitle

\section{Project Summary/Abstract} % 30 lines of text

\documentclass[12pt]{article}
\usepackage{amsmath}
\usepackage[letterpaper,margin=.5in]{geometry}
\usepackage{url}
\usepackage{graphicx}
\usepackage{helvet}
\renewcommand{\familydefault}{\sfdefault}
\pagenumbering{gobble}

\title{Analysis of Somatic Mutations in Mice}
\date{}
% \date{\today}
% \author{Adam Orr}

\begin{document}
\maketitle

% 250 words

% The abstract should include:
%  * a brief background of the project;
%  * specific aims, objectives, or hypotheses;
%  * the significance of the proposed research and relevance to public health;
%  * the unique features and innovation of the project;
%  * the methodology (action steps) to be used;
%  * expected results; and
%  * description of how your results will affect other research areas.
% Suggestions
%  * Be complete, but brief.
%  * Use all the space allotted.
%  * Avoid describing past accomplishments and the use of the first person.
%  * Write the abstract last so that it reflects the entire application.



\section{Abstract}

Cancer is a devastating disease that affects millions. The disease is caused by mutations in somatic cells that cause those cells to rapidly proliferate and evade the immune system. Despite this fact, little is known about somatic mutations themselves, particularly in mammals. 

\end{document}

\clearpage

\section{Specific Aims} %1 page

%1 page
% State concisely the goals of the proposed research and summarize the expected outcome(s),
% including the impact that the results of the proposed research will have on the research field(s)
% involved.
% List succinctly the specific objectives of the research proposed (e.g., to test a stated hypothesis,
% create a novel design, solve a specific problem, challenge an existing paradigm or clinical practice,
% address a critical barrier to progress in the field, or develop new technology)

Somatic mutations are mutations that occur in the soma.
Though they occur infrequently, the size of mammalian genomes suggest approximately one somatic mutation occurs every time a cell divides.
As these mutations are not heritable, they usually do not impact the organism harboring these mutations.
However, the somatic mutation theory of cancer suggests that somatic cells accumulate mutations until a sufficient number of mutations in oncogenes or tumor suppressor genes cause the mutated cell to become cancerous.
Thus, cancer incidence can be impacted by increasing the rate of mutation \textit{or} increasing the probability of mutation in an oncogene or tumor suppressor gene.
While the role of nonspecific mutagens in oncogenesis has been thoroughly studied, the role of factors that could change the distribution of somatic mutations has not.

%talk about clues in chromatin association and differences in tissue chromatin str.
Recent work has revealed that the locations of mutations in cancer tumors are associated with regions of high chromatin accessibility.
To fit in the nucleus, eukaryotic DNA is bound to histones and packaged into chromatin.
The DNA may be in a loosely bound, high accessibility state to allow DNA-binding factors to access the DNA.
The DNA may also be in a tightly bound, low accessibility state that restricts the ability of proteins to access the DNA.
Levels of chromatin accessibility vary with levels of gene regulation; low accessibility is associated with repressed genes and high accessibility is associated with actively expressed genes.
One of the most important sources of variability in chromatin structure is tissue differentiation.
Different tissues express particular genes and repress others by decreasing DNA accessibility.
If chromatin accessibility affects the location of mutations, and locations with high levels of accessibility vary by tissue, the locations of somatic mutations could vary in different tissues.
If there is variation in chromatin accessibility at oncogenes and tumor suppressor genes, this variation could cause differential risk of cancer for different tissues.
This hypothesis is consistent with the observation that different tissues have different rates of cancer.
However, it is still unclear whether these associations form only after the tissue becomes cancerous. 

A significant impediment to research in this area is a lack of computational methods for detecting somatic mutations in healthy tissue.
Current methods for detecting somatic mutation focus on detecting somatic mutations present in tumors, and rely on data from matched normal tissue to improve accuracy.
Cancer genotypes are significantly altered compared to their normal counterparts, and are characterized by large genomic aberrations such as chromosome rearrangements, changes in ploidy number, inversions, and insertion/deletions.
On the other hand, mutations in healthy somatic tissue are more subtle and are more likely to contain single nucleotide changes and short changes in microsatellite length.
At the same time, sequencing errors most likely appear as single nucleotide polymorphisms or short indels---indistinguishable from the types of mutations most common in healthy somatic tissue.
Additionally, the sequencing error rate is significantly higher than the somatic mutation rate of healthy cells while being lower than mutation rates of cancer tumors.
Due to the significant differences in the type and number of somatic mutations between healthy tissue and cancer tumors, current methods are insufficient for detecting somatic mutations in healthy tissue.

\textbf{Aim 1 of this project is to develop a novel bioinformatic pipeline to detect somatic mutations and estimate chromatin accessibility across the genome in somatic samples.}
This will be done by modeling and correcting sequencing errors in Next-Generation Sequencing reads to reduce erroneous genotype calls. Genotypes will then be called using model parameters that closely resemble those expected in healthy somatic tissue.
Additionally, reads generated by an ATAC-seq or similar sequence-based accessibility assay will incorporated into the alignment to improve error correction and genotype calling while providing information on chromatin accessibility.

\textbf{Aim 2 of this project is to test the hypothesis that chromatin accessibility significantly impacts mutation rate.}
The method developed in Aim 1 will be applied to several somatic samples from two mouse models of cancer.
As a control, it will also be applied to samples of the same tissues in a mouse with a similar, unmanipulated genetic background.
This will provide an estimate of the distribution of chromatin accessibility values at mutated sites; if this distribution is not significantly different between different tissues, the hypothesis is supported.

This project will provide a novel method for analyzing somatic mutations in healthy tissues and fill a knowledge gap regarding the relationship between chromatin structure and mutation rate. It will provide insight into an important chicken-and-egg problem in cancer genomics and provide an important resource for the fields of developmental biology and epigenomics.


\clearpage

\section{Research Strategy} %6 pages
\subsection{Significance}

% why study somatic mutation?
Somatic mutations are mutations that occur in somatic cells, and as such are not passed to offspring.
These mutations arise due to errors in DNA replication preceding cell division.
While the DNA replication and error repair machinery is highly accurate, mutations occur at high enough frequencies to be significant.
Current assumptions of error rate and genome size imply that on average each cell division results in the accumulation of one novel mutation in the daughter cells.

These mutations are central to the somatic mutation theory of cancer.
This theory posits that cancers arise in multicellular organisms when somatic cells of that organism accumulate a sufficient number of mutations in oncogenes or tumor suppressor genes induces tumorigenesis.
Important mutations in oncogenes cause them to be over activated, promoting unnecessary cell proliferation.
In contrast, relevant mutations in tumor suppressor genes cause them to lose functionality, inhibiting their ability to limit cell proliferation.
Once a sufficient number of these mutations accumulate, the cell divides rapidly and accumulates even more somatic mutations, severely destabilizing the genome.
Thus, understanding how somatic mutations accumulate in healthy cells is vital to understanding the onset and progression of cancer. 

%why study chromatin accessibility?
A similarly important factor required for the onset of cancer is the deregulation of chromatin.
The chromatin state in a region of the genome is determined by the organization of the nucleosomes in the region.
These nucleosomes are composed of histones, which bind and organize a segment of DNA.
While the function of higher orders of organization is not well understood, the function and impact of different chromatin states is clear.
Chromatin takes two important possible states: heterochromatin, in which the DNA is tightly packaged; and euchromatin, in which the DNA is loosely packaged.
In the heterochromatin state, translation is repressed as the DNA is inaccessible to translation proteins.
In the euchromatin state, translation can occur as the DNA is available for translation complexes to bind.

As the chromatin state significantly impacts translational ability, and different tissues must express different genes to maintain their identity, different tissues have different distributions of chromatin states.
The process of switching between these states is known as chromatin remodeling.
As cancers require normally silenced genes to be expressed, misregulated chromatin remodeling is important for tumorigenesis.
In fact, many genes implicated as oncogenes or tumor suppressor genes have functions important for chromatin remodeling.
As such, understanding how these epigenetic changes cause disease is vital to understanding and preventing cancer.


% why study the interface of somatic mutation and chromatin accessibility?
Thus, it is apparent that a comprehensive understanding of cancer onset requires integrating both somatic mutation and epigenetic misregulation.
The somatic mutation theory of cancer can explain both components, as mutations in chromatin remodeling proteins can cause the large-scale changes in expression required for tumors to form and thrive.
Some critics claim that the theory inadequately explains how the requisite mutations are obtained.
As somatic mutations occur at very low frequency, the probability of accumulating the number of required mutations in the correct sites and within the same cell is low.
A potential resolution to this inconsistency is the observation that mutation rates are not evenly distributed across the genome.
That is, mutation rates could be high enough in regions that contain oncogenes and tumor suppressor genes that mutating a sufficient number of them can occur with probabilities consistent with rates of cancer incidence.

Recently, many groups have reported a strong association between regional variation in somatic mutation rates in tumors and chromatin states in that region.
This pattern has also been seen in germline mutations.
Thus, it is highly likely that chromatin states impact local mutation rates in healthy somatic tissues as well.
Because chromatin remodeling is an important aspect of cell differentiation, different tissues have different patterns of chromatin states.
This implies that different tissues have different distributions of mutation rates.

One observation that indicates that this difference may be significant for studying cancer is that different tissues have different rates of cancer incidence.
There is debate about how much of this difference can be attributed to difference in cell division rates between different tissues.
However, differences in chromatin states of oncogenes and tumor suppressor genes in healthy somatic tissue may play an important role.

% Aim 1 of this project is to develop a novel bioinformatic pipeline to detect somatic mutations and estimate chromatin accessibility across the genome in somatic samples.
\subsubsection{Aim 1}

% why do we need a method to jointly detect somatic mutations and chromatin accessibility?
Despite the central role of somatic mutations in cancer development, they remain difficult to study.
However, if a relationship between chromatin accessibility and mutation rate exists, this can be explicitly modeled to improve detection of somatic mutations.
At the same time, sequencing-based assays of chromatin accessibility hold underlying genotype information.
Thus, considering both types of data at the same time and integrating information from both data sources into a model will increase the ability of the researcher to sensitively and accurately detect somatic mutations.

% how will this method change the field if it is achieved?
% reduce sequencing costs - less covg necessary initially, doesn't throw away  information from ATAC-seq
% 

%%%%%%%%%%%%% aim 2 %%%%%%%%%%%%%%%%

% Aim 2 of this project is to test the hypothesis that chromatin accessibility significantly impacts mutation rate.
\subsubsection{Aim 2}

% why study the interaction of somatic mutation and chromatin accessibility? 

% how will this aim change the field if it is achieved?

% 1. Significance
% Explain the importance of the problem or critical barrier to progress that the proposed project addresses.
% Explain how the proposed project will improve scientific knowledge, technical capability, and/or clinical practice in one or more broad fields.
% Describe how the concepts, methods, technologies, treatments, services, or preventative interventions that drive this field will be changed if the proposed aims are achieved.

\subsection{Innovation}
% Explain how the application challenges and seeks to shift current research or clinical practice paradigms.
% Describe any novel theoretical concepts, approaches or methodologies, instrumentation or interventions to be developed or used, and any advantage over existing methodologies, instrumentation, or interventions.
% Explain any refinements, improvements, or new applications of theoretical concepts, approaches or methodologies, instrumentation, or interventions.

%

\subsubsection{Aim 1}

While there has been a large body of work on the generation of somatic mutations in the presence of mutagens, much less is known about the endogenous process that generates somatic mutations--that is, the process that generates somatic mutations even in cells that are never exposed to mutagens or unusual stress.
% Despite the critical role somatic mutations play in tumorigenesis, these mutations are usually studied only in tumors, as there are a sufficient number of mutations that a significant number can be detected at modest sequencing depth.
% However, most of these mutations did not cause the tumor, and occur only after the tumor has formed.
This impedes researchers' ability to detect significant departures from healthy levels and patterns of somatic mutation; a significant amount of energy is expended in differentiating non-pathogenic "passenger" mutations from the "driver" mutations that actually contribute to tumor formation.
This knowledge gap is largely due to the technical difficulty of detecting somatic mutations in healthy tissues.
In any sample of healthy tissue, the number of somatic mutations within the sample is expected to be orders of magnitude smaller than the number of sequencing errors.
A reliable method of detecting somatic mutations in healthy cells and non-tumor samples will enable researchers to study somatic mutations in healthy tissue, improving their ability to distinguish between natural patterns of somatic variation and unhealthy levels of somatic variation indicative of disease.

% take cancer biology from purely historical to active 

% if mutations coincide with chromatin accessibility, that can be modeled.

% what's new about software that detects somatic mutations and estimates chromatin accessibility?

\subsubsection{Aim 2}

% what's new about considering somatic mutations along with chromatin accessibility?

\subsection{Approach}
\subsubsection{Aim 1}
\subsubsection{Aim 2}
% 2. Approach
% Describe the overall strategy, methodology, and analyses to be used to accomplish the specific aims of the project. Unless addressed separately in the Resource Sharing Plan attachment, include how the data will be collected, analyzed, and interpreted as well as any resource sharing plans as appropriate.
% Discuss potential problems, alternative strategies, and benchmarks for success anticipated to achieve the aims.
% If the project is in the early stages of development, describe any strategy to establish feasibility, and address the management of any high risk aspects of the proposed work.
% Point out any procedures, situations, or materials that may be hazardous to personnel and the precautions to be exercised. A full discussion on the use of select agents should appear in the Select Agent Research attachment below.

% If you have multiple Specific Aims, you may address Significance, Innovation, and Approach either for each Specific Aim individually or for all of the Specific Aims collectively. As applicable, also include the following information as part of the Research Strategy, keeping within the three sections (Significance, Innovation, and Approach) listed above.




\section{References} %no limit

\end{document}