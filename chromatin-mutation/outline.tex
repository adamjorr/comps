% !TEX program=lualatex
\RequirePackage{luatex85}
\documentclass[11pt]{article}
\usepackage{amsmath}
\usepackage[letterpaper,margin=.5in]{geometry}
\usepackage{url}
\usepackage{graphicx}
\usepackage[utf8]{inputenc}
\usepackage[T1]{fontenc}
\usepackage{textcomp}
\usepackage[scaled]{helvet}
\usepackage[compact,tiny]{titlesec}
\usepackage[style=nature, citestyle=authoryear]{biblatex}
\usepackage{tikz}
\usepackage{wrapfig}
\renewcommand{\familydefault}{\sfdefault}
\pagenumbering{gobble}
\bibliography{comps.bib}

\usepackage{outline}

% \item bar \parencite{zhang_extra_2011}

\begin{document}

% • What do you intend to do?
% • Why is this worth doing or the significance of the research? How is it innovative?
% • What has already been done in general, and what have other researchers done
% in this field? Use appropriate references. What will this new work add to the field
% of knowledge?
% • What have you (and your collaborators) done to establish the feasibility of what
% you are proposing to do?
% • How will the research be accomplished? Who? What? When? Where? Why?

% 1. Make sure that all sections are internally consistent and that they dovetail with each
% other. Use a numbering system, and make sections easy to find. Lead the reviewers
% through your research plan. One person should revise and edit the final draft.
% 2. Show knowledge of recent literature and explain how the proposed research will
% further what is already known.
% 3. Emphasize how some combination of a novel hypothesis, important preliminary
% data, a new experimental system and/or a new experimental approach will enable
% important progress to be made.
% 4. Establish credibility of the proposed principal investigator and the collaborating
% researchers. 

% 12 pages
% Significance : Recommended Length: Approximately 1-2 pages 
% Innovation : Recommended Length: The recommended length of the innovation section is 1/2-1
% Approach : Recommended Length: The maximum recommended length of the approach section is 9-10 pages. 

% 6 pages
% Significance : Recommended Length: Approximately 0.5-1 pages 
% Innovation : Recommended Length: The recommended length of the innovation section is 0.25-0.5
% Approach : Recommended Length: The maximum recommended length of the approach section is 4.5-5 pages. 

% Aim 1 - develop a novel bioinformatic pipeline to detect somatic mutations and estimate chromatin accessibility across the genome
% Aim 2 - test the hypothesis that chromatin accessibility significantly impacts mutation rate
\begin{outline}
	\item Significance
	\begin{outline}
		\item Somatic mutations play an important role in carcinogenesis
		\begin{outline}
			\item Genome instability is a hallmark of cancer \parencite{hanahan_hallmarks_2000, hanahan_hallmarks_2011}
			\item The Somatic Mutation Theory of cancer suggests that somatic mutations slowly accumulate until a sufficient number disrupt normal cell cycle control \parencite{tomasetti_variation_2015}
			\item There is disagreement on the role somatic mutations play in carcinogenesis \parencite{baker_cancer_2015}
			\begin{outline}
				\item The Somatic Mutation Theory doesn't explain how a sufficient number of somatic mutations accumulate \parencite{baker_cancer_2015}
				\item Variation in somatic tissue division rates explains a large proportion of variation in incidence by tissue \parencite{tomasetti_variation_2015} implying cancer is caused by mutations that occur during cell division, but there is disagreement \parencite{rozhok_critical_2015, wang_implications_2015}
				\item Our understanding of the rate of somatic mutations, and how that changes in different types of cells, is limited \parencite{stratton_cancer_2009}
			\end{outline}
		\end{outline}
		\item The role of chromatin states in somatic mutation
		\item Measuring chromatin states
		\begin{outline}
			\item Sequence Characteristics
			\item Next-Gen Sequencing Assays
			\begin{outline}
				\item ATAC-seq
			\end{outline}
		\end{outline}
		\item Mutation Types Associated with Chromatin States
	\end{outline}
	\item Innovation
	\begin{outline}
		\item foo
	\end{outline}
	\item Approach
	\begin{outline}
		\item Aim 1 - develop a novel bioinformatic pipeline to detect somatic mutations and estimate chromatin accessibility across the genome
			\begin{outline}
				\item Align
				\item Alternative Approeaches
			\end{outline}
		\item Aim 2 - test the hypothesis that chromatin accessibility significantly impacts mutation rate
			\begin{outline}
				\item Sample Sources
				\item Tissue collection
				\item Sequencing
				\item ATAC-seq
				\item Statistical Analysis of Results
				\item Alternative Approaches
			\end{outline}
	\end{outline}
\end{outline}
\medskip

\printbibliography






\end{document}







