% !TEX program=lualatex
\RequirePackage{luatex85}
\documentclass[11pt]{article}
\usepackage{amsmath}
\usepackage[letterpaper,margin=.5in]{geometry}
\usepackage{url}
\usepackage{graphicx}
\usepackage[utf8]{inputenc}
\usepackage[T1]{fontenc}
\usepackage{textcomp}
\usepackage[scaled]{helvet}
\usepackage[compact,tiny]{titlesec}
\usepackage[style=nature, citestyle=authoryear]{biblatex}
\usepackage{tikz}
\usepackage{wrapfig}
\renewcommand{\familydefault}{\sfdefault}
\pagenumbering{gobble}
\bibliography{comps.bib}

\usepackage{outline}

% \item bar \parencite{zhang_extra_2011}

\begin{document}

% • What do you intend to do?
% • Why is this worth doing or the significance of the research? How is it innovative?
% • What has already been done in general, and what have other researchers done
% in this field? Use appropriate references. What will this new work add to the field
% of knowledge?
% • What have you (and your collaborators) done to establish the feasibility of what
% you are proposing to do?
% • How will the research be accomplished? Who? What? When? Where? Why?

% 1. Make sure that all sections are internally consistent and that they dovetail with each
% other. Use a numbering system, and make sections easy to find. Lead the reviewers
% through your research plan. One person should revise and edit the final draft.
% 2. Show knowledge of recent literature and explain how the proposed research will
% further what is already known.
% 3. Emphasize how some combination of a novel hypothesis, important preliminary
% data, a new experimental system and/or a new experimental approach will enable
% important progress to be made.
% 4. Establish credibility of the proposed principal investigator and the collaborating
% researchers. 

% 12 pages
% Significance : Recommended Length: Approximately 1-2 pages 
% Innovation : Recommended Length: The recommended length of the innovation section is 1/2-1
% Approach : Recommended Length: The maximum recommended length of the approach section is 9-10 pages. 

% 6 pages
% Significance : Recommended Length: Approximately 0.5-1 pages 
% Innovation : Recommended Length: The recommended length of the innovation section is 0.25-0.5
% Approach : Recommended Length: The maximum recommended length of the approach section is 4.5-5 pages. 

\begin{outline}
\item Aim 1 - develop a novel bioinformatic pipeline to detect somatic mutations and estimate chromatin accessibility across the genome
	\begin{outline}
		\item Significance
		\item Innovation
		\item Approach
	\end{outline}
\item Aim 2 - test the hypothesis that chromatin accessibility significantly impacts mutation rate
	\begin{outline}
		\item Significance
		\item Innovation
		\item Approach
		\begin{outline}
			\item Data
		\end{outline}
	\end{outline}
\end{outline}

\medskip

\printbibliography






\end{document}







