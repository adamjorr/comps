\documentclass[12pt]{article}
\usepackage{amsmath}
\usepackage[letterpaper,margin=.5in]{geometry}
\usepackage{url}
\usepackage{graphicx}
\usepackage[utf8]{inputenc}
\usepackage[T1]{fontenc}
\usepackage{textcomp}
\usepackage[scaled]{helvet}
\renewcommand{\familydefault}{\sfdefault}
\pagenumbering{gobble}

\title{Analysis of Somatic Mutations In Vegetatively Reproducing Grapes}
\date{}
% \date{\today}
% \author{Adam Orr}

\begin{document}
\maketitle

% 250 words

% The abstract should include:
%  * a brief background of the project;
%  * specific aims, objectives, or hypotheses;
%  * the significance of the proposed research and relevance to public health;
%  * the unique features and innovation of the project;
%  * the methodology (action steps) to be used;
%  * expected results; and
%  * description of how your results will affect other research areas.
% Suggestions
%  * Be complete, but brief.
%  * Use all the space allotted.
%  * Avoid describing past accomplishments and the use of the first person.
%  * Write the abstract last so that it reflects the entire application.



\section{Abstract}
Cancer is a devastating disease caused by mutations in somatic cells that cause those cells to rapidly proliferate and evade the immune system. Despite this fact, little is known about somatic mutations themselves. Of particular interest are the differences between mutation rates and profiles in eukaryotes that reproduce sexually and those that reproduce by fragmentation of the soma, similar to the mode of replication of cancer cells.
\textbf{Aim 1} of this project is to test the hypothesis that the somatic mutation rate of a population of outcrossing grapes will be higher than the somatic mutation rate from a population of grapes that solely reproduce vegetatively. This hypothesis will be tested by sequencing three leaves in triplicate from grapes from a population of wild grapes matched with those from a population of domestic, vegetatively reproducing grapes. 




\end{document}