\documentclass[12pt]{article}
\usepackage{amsmath}
\usepackage[letterpaper,margin=.5in]{geometry}
\usepackage{url}
\usepackage{graphicx}
\usepackage[utf8]{inputenc}
\usepackage[T1]{fontenc}
\usepackage{textcomp}
\usepackage[scaled]{helvet}
\renewcommand{\familydefault}{\sfdefault}
\pagenumbering{gobble}

\title{Calculating Somatic Mutation Rates in Different Mice Tissues}
\date{}
% \date{\today}
% \author{Adam Orr}

\begin{document}
\maketitle

% 250 words

% The abstract should include:
%  * a brief background of the project;
%  * specific aims, objectives, or hypotheses;
%  * the significance of the proposed research and relevance to public health;
%  * the unique features and innovation of the project;
%  * the methodology (action steps) to be used;
%  * expected results; and
%  * description of how your results will affect other research areas.
% Suggestions
%  * Be complete, but brief.
%  * Use all the space allotted.
%  * Avoid describing past accomplishments and the use of the first person.
%  * Write the abstract last so that it reflects the entire application.



\section{Abstract}

Cancer is a devastating disease caused by mutations in somatic cells that cause those cells to rapidly proliferate and evade the immune system.
Despite this fact, little is known about somatic mutations themselves, particularly in mammals.
One reason somatic mutations have been difficult to study is that they are exceedingly rare, making them difficult to confidently detect.
This problem is exacerbated by sequencing error rates that are orders of magnitude higher than somatic mutation rates.
However, recent advances in sequencing technology and computational methods make such a problem tractable.
The proposed project will elucidate how somatic mutations are produced and spread in normal mice and in a cancer model of mice.
\textbf{Aim 1} of the project is to test the hypothesis that tissue proliferation rate is positively correlated with somatic mutation rate.
To test this hypothesis, various mouse tissue will be collected and sequenced in triplicate, and somatic variants and the somatic mutation rate will be determined by a novel bioinformatic pipeline.
The somatic mutation rates of each tissue will then be compared to the tissue proliferation rate.
\textbf{Aim 2} of the project is to test the hypothesis that tissues destined to become cancerous have higher somatic mutation rates than similar noncancerous tissues.
To test this hypothesis, the somatic mutation rates calculated for Aim 1 for each tissue will be compared between the different genetic backgrounds of mice.
These results will improve the understanding of how somatic mutations arise and will impact the work of cancer and developmental biologists.


\end{document}